% !TeX root = exercise02.tex
\documentclass[a4paper,11pt]{article}
\usepackage{a4wide}
\usepackage{graphicx}
\usepackage[T1]{fontenc}
\usepackage[latin1]{inputenc}
%\usepackage[utf8]{inputenc}
\usepackage{ae}
\usepackage{url}
\usepackage{enumerate}
\usepackage{float}
\usepackage{amsmath}
\usepackage{subfigure}
\usepackage{color}
\usepackage{listings}
\usepackage[final]{pdfpages}
\usepackage{floatflt}
\usepackage{booktabs}
\usepackage{epstopdf}
\usepackage{graphicx, array, blindtext}
\usepackage{fancyhdr}
\usepackage{hyperref}
\usepackage{enumitem}
\usepackage{xcolor}
\usepackage{fancyvrb}

\font\helvSmallMediumFont=phvr at 4.0mm
\font\helvSmallBoldFont=phvb at 4.0mm
\font\helvMediumFont=phvr at 4.3mm
\font\helvBoldFont=phvb at 4.3mm
\font\ncsLargeFont=pncr at 5.0mm
\def\helvSmall{\helvSmallBoldFont\baselineskip=3mm}
\def\helvMedium{\helvMediumFont\baselineskip=6mm}
\def\ncsLarge{\ncsLargeFont\baselineskip=21mm}

\sloppy

\pagestyle{empty}

\definecolor{mygreen}{rgb}{0,0.6,0}
\definecolor{mygray}{rgb}{0.5,0.5,0.5}
\definecolor{mymauve}{rgb}{0.58,0,0.82}

\lstset{
  basicstyle=\color{white}\ttfamily,
  breaklines=true,
  backgroundcolor=\color{black},
  frame=none,
  columns=fullflexible
}

\setlist{noitemsep,topsep=5pt,parsep=0pt,partopsep=0pt}

\setlength{\parindent}{0px}

\begin{document}
\newcommand{\dozenten}{Prof.~Dr.~Steffen Herbold}
\newcommand{\vorlesung}{Principles of AI Engineering}
\newcommand{\docauthor}{Lukas Schulte}
\newcommand{\semester}{}
\newcommand{\blattnummer}{2}
\newcommand{\bistermin}{}
\addtolength{\topmargin}{-2cm}
\noindent

\begin{table}[ht]
    \centering
    \begin{tabular}{*{2}{m{0.48\textwidth}}}
        \helvBoldFont \vorlesung \helvMedium


        \vspace{1ex}
        \helvBoldFont Exercise \blattnummer


        \vspace{1ex}
        \dozenten


        \helvMedium\semester

         &
        \includegraphics[width=0.45\textwidth]{common/aie_square.png}
    \end{tabular}
\end{table}
\vspace{-1.75cm}
\hrule width \columnwidth
\begin{flushright}
    \vspace{-0.25cm}
    \small Author: \docauthor
    \vspace{-0.75cm}
\end{flushright}


\section*{Project task}
% Analyzing the data and pre-processing

Your coworker provided a first dataset, which you can use to train your AI model (\texttt{sample1.csv.gz} in StudIP). Load the data into a Jupyter Notebook, inspect it and implement appropriate text pre-processing steps. Collect at least one example which shows what each of your pre-processing steps does.

\vspace{5px}

Hint:
Next week's exercise will deal with creating the model. The model (a random forest model) will be built with \href{https://scikit-learn.org}{sklearn}\footnote{\url{https://scikit-learn.org}}.

\section*{Questions}

\begin{enumerate}
      \item
            Analyze the dataset.
            \begin{itemize}
                  \item Which fields are relevant for the prediction in the given project scenario?
                  \item Is the dataset even suitable for the task at hand? Do you spot potential problems?
            \end{itemize}
      \item
            Define the following text pre-processing steps:
            \begin{itemize}
                  \item Tokenization
                  \item Normalization
                  \item Noise removal
                  \item Stemming
                  \item Lemmatization
                  \item Stop-word removal
            \end{itemize}
      \item
            Explain concept drift and how it would impact your application.
            Describe how you would mitigate the effects.
      \item
            Describe what could be considered a feedback loop in your application and why that can be a problem.
            Explain how you can measure the effect.
\end{enumerate}

\end{document}
