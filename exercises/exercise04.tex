% !TeX root = exercise04.tex
\documentclass[a4paper,11pt]{article}
\usepackage{a4wide}
\usepackage{graphicx}
\usepackage[T1]{fontenc}
\usepackage[latin1]{inputenc}
%\usepackage[utf8]{inputenc}
\usepackage{ae}
\usepackage{url}
\usepackage{enumerate}
\usepackage{float}
\usepackage{amsmath}
\usepackage{subfigure}
\usepackage{color}
\usepackage{listings}
\usepackage[final]{pdfpages}
\usepackage{floatflt}
\usepackage{booktabs}
\usepackage{epstopdf}
\usepackage{graphicx, array, blindtext}
\usepackage{fancyhdr}
\usepackage{hyperref}
\usepackage{enumitem}
\usepackage{xcolor}
\usepackage{fancyvrb}

\font\helvSmallMediumFont=phvr at 4.0mm
\font\helvSmallBoldFont=phvb at 4.0mm
\font\helvMediumFont=phvr at 4.3mm
\font\helvBoldFont=phvb at 4.3mm
\font\ncsLargeFont=pncr at 5.0mm
\def\helvSmall{\helvSmallBoldFont\baselineskip=3mm}
\def\helvMedium{\helvMediumFont\baselineskip=6mm}
\def\ncsLarge{\ncsLargeFont\baselineskip=21mm}

\sloppy

\pagestyle{empty}

\definecolor{mygreen}{rgb}{0,0.6,0}
\definecolor{mygray}{rgb}{0.5,0.5,0.5}
\definecolor{mymauve}{rgb}{0.58,0,0.82}

\lstset{
  basicstyle=\color{white}\ttfamily,
  breaklines=true,
  backgroundcolor=\color{black},
  frame=none,
  columns=fullflexible
}

\setlist{noitemsep,topsep=5pt,parsep=0pt,partopsep=0pt}

\setlength{\parindent}{0px}

\begin{document}
\newcommand{\dozenten}{Prof.~Dr.~Steffen Herbold}
\newcommand{\vorlesung}{Principles of AI Engineering}
\newcommand{\docauthor}{Lukas Schulte}
\newcommand{\semester}{}
\newcommand{\blattnummer}{4}
\newcommand{\bistermin}{}
\addtolength{\topmargin}{-2cm}
\noindent

\begin{table}[ht]
    \centering
    \begin{tabular}{*{2}{m{0.48\textwidth}}}
        \helvBoldFont \vorlesung \helvMedium


        \vspace{1ex}
        \helvBoldFont Exercise \blattnummer


        \vspace{1ex}
        \dozenten


        \helvMedium\semester

         &
        \includegraphics[width=0.45\textwidth]{common/aie_square.png}
    \end{tabular}
\end{table}
\vspace{-1.75cm}
\hrule width \columnwidth
\begin{flushright}
    \vspace{-0.25cm}
    \small Author: \docauthor
    \vspace{-0.75cm}
\end{flushright}


\section*{Project task}
% API and documentation

In the last exercises, you created a model and modular pre-processing steps. Now it is time to start working on the application that will enable our users to predict the types of issues. That application will consist of a frontend and a backend. This week, you will create a REST API for the backend.

\vspace{5px}

Create a FLASK app that offers endpoints implementing the specifications below. It is sufficient if the endpoints return dummy data for now. Test the API with an API testing tool like \href{https://postman.com}{Postman}\footnote{\url{https://postman.com}} or \href{https://hoppscotch.io}{Hoppscotch}\footnote{\url{https://hoppscotch.io}}.

\section*{Questions}

\begin{enumerate}
      \item
            Given the problem description, describe how flawed predictions can be minimized and how you can represent this in a graphical user interface within the application.
            You can assume that you have some indication of whether the model is sure or not about the prediction.
            You can also assume a case where you know the input, e.g., the language, and whether the model works well or not well with the language.
      \item
            Given the problem description, does the application use deductive or inductive reasoning, describe which and why.
            Describe, what would be different in the application if it would be the other way around, e.g., different model or no model at all.
            Briefly explain if the application can still fulfill its purpose and to what extent.
\end{enumerate}

\newpage
\section{API specification}

\subsection*{Predict issue type}

\texttt{{[}POST{]}} - \texttt{\textbackslash{}api\textbackslash{}predict}

\subsubsection*{Body parameters}

\begin{tabular*}{\linewidth}[t]{@{}lll@{}} \toprule
      Parameter      & Type            & Description                     \\ \midrule
      \texttt{title} & \texttt{string} & Title of the issue              \\
      \texttt{body}  & \texttt{string} & Body / description of the issue \\ \bottomrule
\end{tabular*}

\subsubsection*{Response parameters}


\begin{tabular*}{\linewidth}[t]{@{}lll@{}} \toprule
      Parameter      & Type            & Description     \\ \midrule
      \texttt{id}    & \texttt{string} & ID of the issue \\
      \texttt{label} & \texttt{string} & Predicted label \\ \bottomrule
\end{tabular*}

\subsection*{Correct issue type}

\texttt{{[}POST{]}} - \texttt{\textbackslash{}api\textbackslash{}correct}

\subsubsection*{Body parameters}

\begin{tabular*}{\linewidth}[t]{@{}lll@{}} \toprule
      Parameter      & Type            & Description     \\ \midrule
      \texttt{id}    & \texttt{string} & ID of the issue \\
      \texttt{label} & \texttt{string} & Corrected label \\ \bottomrule
\end{tabular*}

\subsubsection*{Response parameters}


\begin{tabular*}{\linewidth}[t]{@{}lll@{}}
      \toprule
      Parameter   & Type            & Description     \\ \midrule
      \texttt{id} & \texttt{string} & ID of the issue \\ \bottomrule
\end{tabular*}
\vspace{15px}



\end{document}
