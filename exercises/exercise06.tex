% !TeX root = exercise06.tex
\documentclass[a4paper,11pt]{article}
\usepackage{a4wide}
\usepackage{graphicx}
\usepackage[T1]{fontenc}
\usepackage[latin1]{inputenc}
%\usepackage[utf8]{inputenc}
\usepackage{ae}
\usepackage{url}
\usepackage{enumerate}
\usepackage{float}
\usepackage{amsmath}
\usepackage{subfigure}
\usepackage{color}
\usepackage{listings}
\usepackage[final]{pdfpages}
\usepackage{floatflt}
\usepackage{booktabs}
\usepackage{epstopdf}
\usepackage{graphicx, array, blindtext}
\usepackage{fancyhdr}
\usepackage{hyperref}
\usepackage{enumitem}
\usepackage{xcolor}
\usepackage{fancyvrb}

\font\helvSmallMediumFont=phvr at 4.0mm
\font\helvSmallBoldFont=phvb at 4.0mm
\font\helvMediumFont=phvr at 4.3mm
\font\helvBoldFont=phvb at 4.3mm
\font\ncsLargeFont=pncr at 5.0mm
\def\helvSmall{\helvSmallBoldFont\baselineskip=3mm}
\def\helvMedium{\helvMediumFont\baselineskip=6mm}
\def\ncsLarge{\ncsLargeFont\baselineskip=21mm}

\sloppy

\pagestyle{empty}

\definecolor{mygreen}{rgb}{0,0.6,0}
\definecolor{mygray}{rgb}{0.5,0.5,0.5}
\definecolor{mymauve}{rgb}{0.58,0,0.82}

\lstset{
  basicstyle=\color{white}\ttfamily,
  breaklines=true,
  backgroundcolor=\color{black},
  frame=none,
  columns=fullflexible
}

\setlist{noitemsep,topsep=5pt,parsep=0pt,partopsep=0pt}

\setlength{\parindent}{0px}

\begin{document}
\newcommand{\dozenten}{Prof.~Dr.~Steffen Herbold}
\newcommand{\vorlesung}{Principles of AI Engineering}
\newcommand{\docauthor}{Lukas Schulte}
\newcommand{\semester}{}
\newcommand{\blattnummer}{6}
\newcommand{\bistermin}{}
\addtolength{\topmargin}{-2cm}
\noindent

\begin{table}[ht]
    \centering
    \begin{tabular}{*{2}{m{0.48\textwidth}}}
        \helvBoldFont \vorlesung \helvMedium


        \vspace{1ex}
        \helvBoldFont Exercise \blattnummer


        \vspace{1ex}
        \dozenten


        \helvMedium\semester

         &
        \includegraphics[width=0.45\textwidth]{common/aie_square.png}
    \end{tabular}
\end{table}
\vspace{-1.75cm}
\hrule width \columnwidth
\begin{flushright}
    \vspace{-0.25cm}
    \small Author: \docauthor
    \vspace{-0.75cm}
\end{flushright}


\section*{Project task}
% Tests, continuous integration and coverage

Write tests for your applications. Use the Python library \href{https://pytest.org}{pytest}\footnote{\url{https://pytest.org}} to write at least one meaningful test for each endpoint.

\vspace{5px}

Add a \href{https://docs.gitlab.com/ee/ci/quick_start/}{GitLab CI}\footnote{\url{https://docs.gitlab.com/ee/ci/quick_start/}} to your repository that executes your tests automatically. A public GitLab Runner is active for all FIM Git projects.\footnote{\url{https://www.fim.uni-passau.de/en/it-services/online-services/gitlab}} Further include a coverage report using \href{https://github.com/pytest-dev/pytest-cov}{pytest-cov}\footnote{\url{https://github.com/pytest-dev/pytest-cov}}.

\section*{Questions}

\begin{enumerate}
      \item
            Describe how you would test a system that solves the problem description, e.g., unit tests. Consider the weak and strict correctness assumption from the lecture in your answer. Remember that the machine learning model is only one part of the system.
\end{enumerate}

\end{document}
