% !TeX root = exercise08.tex
\documentclass[a4paper,11pt]{article}
\usepackage{a4wide}
\usepackage{graphicx}
\usepackage[T1]{fontenc}
\usepackage[latin1]{inputenc}
%\usepackage[utf8]{inputenc}
\usepackage{ae}
\usepackage{url}
\usepackage{enumerate}
\usepackage{float}
\usepackage{amsmath}
\usepackage{subfigure}
\usepackage{color}
\usepackage{listings}
\usepackage[final]{pdfpages}
\usepackage{floatflt}
\usepackage{booktabs}
\usepackage{epstopdf}
\usepackage{graphicx, array, blindtext}
\usepackage{fancyhdr}
\usepackage{hyperref}
\usepackage{enumitem}
\usepackage{xcolor}
\usepackage{fancyvrb}

\font\helvSmallMediumFont=phvr at 4.0mm
\font\helvSmallBoldFont=phvb at 4.0mm
\font\helvMediumFont=phvr at 4.3mm
\font\helvBoldFont=phvb at 4.3mm
\font\ncsLargeFont=pncr at 5.0mm
\def\helvSmall{\helvSmallBoldFont\baselineskip=3mm}
\def\helvMedium{\helvMediumFont\baselineskip=6mm}
\def\ncsLarge{\ncsLargeFont\baselineskip=21mm}

\sloppy

\pagestyle{empty}

\definecolor{mygreen}{rgb}{0,0.6,0}
\definecolor{mygray}{rgb}{0.5,0.5,0.5}
\definecolor{mymauve}{rgb}{0.58,0,0.82}

\lstset{
  basicstyle=\color{white}\ttfamily,
  breaklines=true,
  backgroundcolor=\color{black},
  frame=none,
  columns=fullflexible
}

\setlist{noitemsep,topsep=5pt,parsep=0pt,partopsep=0pt}

\setlength{\parindent}{0px}

\begin{document}
\newcommand{\dozenten}{Prof.~Dr.~Steffen Herbold}
\newcommand{\vorlesung}{Principles of AI Engineering}
\newcommand{\docauthor}{Lukas Schulte}
\newcommand{\semester}{}
\newcommand{\blattnummer}{8}
\newcommand{\bistermin}{}
\addtolength{\topmargin}{-2cm}
\noindent

\begin{table}[ht]
    \centering
    \begin{tabular}{*{2}{m{0.48\textwidth}}}
        \helvBoldFont \vorlesung \helvMedium


        \vspace{1ex}
        \helvBoldFont Exercise \blattnummer


        \vspace{1ex}
        \dozenten


        \helvMedium\semester

         &
        \includegraphics[width=0.45\textwidth]{common/aie_square.png}
    \end{tabular}
\end{table}
\vspace{-1.75cm}
\hrule width \columnwidth
\begin{flushright}
    \vspace{-0.25cm}
    \small Author: \docauthor
    \vspace{-0.75cm}
\end{flushright}


\section*{Project task}
% Safety assurance

Consider how unsafe input can be handled and how the user might be able to judge the safety of the output. Then create an initial safety box around your model:

\begin{itemize}
      \item
            Check the language of the input and decide what should happen if a user inputs non-English text.
      \item
            Inform the user about the model's confidence during the label correction.
\end{itemize}

\section*{Questions}

\begin{enumerate}
      \item
            Given the project setting, describe the concept of a safety box.
      \item
            Which other measures could be used for safety assurance in the application?
      \item
            Describe the building blocks of fault tree analysis.
      \item
            Describe minimal cut set analysis.
      \item
            Describe failure probability analysis.
\end{enumerate}

\end{document}
